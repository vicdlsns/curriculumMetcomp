% a mashup of hipstercv, friggeri and twenty cv
% https://www.latextemplates.com/template/twenty-seconds-resumecv
% https://www.latextemplates.com/template/friggeri-resume-cv

\documentclass[darkhipster]{simplehipstercv}
% available options are: darkhipster, lighthipster, pastel, allblack, grey, verylight, withoutsidebar
% withoutsidebar
\usepackage[utf8]{inputenc}
\usepackage[default]{raleway}
\usepackage[margin=1cm, a4paper]{geometry}
\usepackage[brazil]{babel}


%------------------------------------------------------------------ Variablen

\newlength{\rightcolwidth}
\newlength{\leftcolwidth}
\setlength{\leftcolwidth}{0.23\textwidth}
\setlength{\rightcolwidth}{0.75\textwidth}

%------------------------------------------------------------------
\title{Curriculo}
\author{\LaTeX{} vdl149}
\date{Junho 2024}

\pagestyle{empty}
\begin{document}


\thispagestyle{empty}
%-------------------------------------------------------------

\section*{Start}

\simpleheader{headercolour}{Victor}{De Luca*}{*Hipotetico 5 anos}{white}



%------------------------------------------------

% this has to be here so the paracols starts..
\subsection*{}
\vspace{4em}

\setlength{\columnsep}{0.5cm}
\columnratio{0.3}[0.6]
\begin{paracol}{2}
\hbadness5000
%\backgroundcolor{c[1]}[rgb]{1,1,0.8} % cream yellow for column-1 %\backgroundcolor{g}[rgb]{0.8,1,1} % \backgroundcolor{l}[rgb]{0,0,0.7} % dark blue for left margin

\paracolbackgroundoptions

% 0.9,0.9,0.9 -- 0.8,0.8,0.8


\footnotesize
{\setasidefontcolour
\flushleft
\begin{center}
    \roundpic{bird.png}
\end{center}

\bg{cvgreen}{white}{Sobre}\\[0.5em]

{\footnotesize
Formado Astronomia Computacional e Astrofísica - UFRJ. Participou do Grupo de Pesquisas Aeroespaciais e participante do Projeto Python em Arquitetura, ambos na Universidade Federal do Rio de Janeiro. Tem experiência na área de Programação e Eletrônica, atuando principalmente nos seguintes temas: foguetes de sondagem atmosférica, sensoriamento remoto, aquisição e processamento de dados, segurança operacional, ensino de programação e integração de linguagens de programação com modelagem e projeto arquitetônico.}
\bigskip

\bg{cvgreen}{white}{Especializações} \\[0.5em]

Programação ~•~ Eletrônica ~•~ Astrofísica ~•~ Aeroespacial

\bigskip


\bg{cvgreen}{white}{Interesses}\\[0.5em]

\texttt{Python} ~/~ \texttt{C++} ~/~ \texttt{Arduino}

\texttt{Astrofísica} ~/~ \texttt{Aeroespacial}

\texttt{Pesquisas}

\vspace{4em}

\infobubble{\faMapMarker}{cvgreen}{white}{Rio de Janeiro - BR}
\infobubble{\faEnvelope}{cvgreen}{white}{victor23@ov.ufrj.br}
\infobubble{\faGithub}{cvgreen}{white}{\href{https://github.com/vicdlsns}{vicdlsns}} 
\infobubble{\faLink}{cvgreen}{white}{\href{https://lattes.cnpq.br/9803655463852943}{Lattes 9803655463852943}}

\phantom{turn the page}

\phantom{turn the page}
}
%-----------------------------------------------------------
\switchcolumn

\small
\section*{Currículo Resumido}

\begin{tabular}{>{\bfseries}r | p{0.5\textwidth} c}
     \cvevent{20xx--20xx}{Estágio - Lorem ipsum}{Algum lugar}{XX \color{cvred}}{\lorem.
    }{bird.png} \\
    \cvevent{2017--20xx}{Extracurricular - Subsistema de Eletrônica de Foguetes}{Minerva Rockets}{POLI/UFRJ \color{cvred}}{Desenvolvimento de sistemas eletrônicos, embarcados e não embarcados, para sensoriamento, instrumentação, controle e monitoramento para uso em pesquisas aeroespaciais. Inclui apresentação dos trabalhos na Semana de Integração Acadêmica (SIAc).
}{ufrj.png} \\
	\cvevent{20xx--20xx}{IC - Astronomia}{Laboratório de X}{OV/UFRJ \color{cvred}}{\lorem. Inclui apresentação dos trabalhos na Semana de Integração Acadêmica (SIAc).
}{ufrj.png} \\
    \cvevent{2020--20xx}{IC - progrsmação para Arquitetura}{Laboratório de Modelos e Fabricação Digital}{FAU/UFRJ \color{cvred}}{Planejamento de uma disciplina de programação mista de visual em Grasshopper e textual em Python voltada a alunos das áreas de arquitetura e urbanismo. Inclui apresentação dos trabalhos na Semana de Integração Acadêmica (SIAc).
}{ufrj.png} \\
    \cvevent{2018--2019}{Monitoria - Programação em Python}{Departamento de Ciência da Computação}{DCC/UFRJ \color{cvred}}{Auxílio ao ensino de programação na linguagem Python.}{ufrj.png} \\
    \cvevent{2014--2016}{IC Jr. - Desenvolvimento de um Sistema Portátil Multiaplicativo de Fluorescência de Raios X}{Laboratório de Instrumentação Eletrônica e Técnicas Analíticas}{FIS/UERJ \color{cvred}}{Desenvolvimento de um sistema multi-aplicativo e das técnicas de medidas para aplicações de espectrometria de raios-x em diversas áreas. Inclui apresentação dos trabalhos no evento UERJ Sem Muros.
}{uerj.png} \\
\end{tabular}

\begin{minipage}[t]{0.33\textwidth}
\section*{Formação Acadêmica}
\begin{tabular}{>{\bfseries}r | p{0.5\textwidth} l}
    \cvdegree{20xx-Atual}{Mestrado em Lorem ipsum}{Em andamento}{UFRJ\color{headerblue}}{}{ufrj.png} \\
    \cvdegree{2023-20xx}{Astronomia Computacional e Astrofísica}{Concluído}{UFRJ\color{headerblue}}{}{ufrj.png} \\
    \cvdegree{2017-2022}{Engenharia de Controle e Automação}{Interrompido}{UFRJ\color{headerblue}}{}{ufrj.png} \\
    \cvdegree{2010-2016}{Fundamental e Médio}{Concluído}{CAp/UERJ\color{headerblue}}{}{uerj.png}
\end{tabular}

\section*{Certificações}
\begin{tabular}{>{\footnotesize\bfseries}r | >{\footnotesize}p{0.55\textwidth}}
    20xx & Primeiro lugar no 1º Campeonato de Piadas Ruins do Morro da Conceição \\
    2020-20xx & Menções Honrosas por trabalhos apresentados - SIAc e JICTAC/UFRJ \\
    2015-2017 & Menções Honrosas e 2° Melhor Trabalho de IC Jr. - Semana de IC/UERJ \\
    2015 & Prata na Olímpiada Brasileira de Física das Escolas Públicas - Estadual
\end{tabular}
\end{minipage}
\hfill
\begin{minipage}[t]{0.33\textwidth}
\section*{Línguas e Idiomas}
\begin{tabular}{l | l}
\textbf{Português} & {\phantom{x}\footnotesize Nativo} \\
\textbf{Inglês} & \pictofraction{\faCircle}{cvgreen}{4}{white!30}{0}{\tiny} \\
\textbf{Francês} & \pictofraction{\faCircle}{cvgreen}{2}{black!30}{2}{\tiny} \\
\textbf{Libras} & \pictofraction{\faCircle}{cvgreen}{2}{black!30}{2}{\tiny} \\
\textbf{Klingon} & \pictofraction{\faCircle}{cvgreen}{2}{black!30}{2}{\tiny} \\
\end{tabular}

\section*{Programação}
\begin{tabular}{r @{\hspace{0.5em}}l}
     \bg{skilllabelcolour}{iconcolour}{Python} &  \barrule{0.55}{0.5em}{cvgreen}\\
     \bg{skilllabelcolour}{iconcolour}{C++/Arduino} & \barrule{0.5}{0.5em}{cvgreen} \\
    \bg{skilllabelcolour}{iconcolour}{Grasshopper} & \barrule{0.4}{0.5em}{cvpurple} \\
     \bg{skilllabelcolour}{iconcolour}{SQL} & \barrule{0.4}{0.5em}{cvpurple} \\
     \bg{skilllabelcolour}{iconcolour}{\LaTeX} & \barrule{0.4}{0.5em}{cvpurple} \\
     \bg{skilllabelcolour}{iconcolour}{Java} & \barrule{0.25}{0.5em}{cvpurple} \\
\end{tabular}

\section*{Publicações e Congressos}
\begin{tabular}{>{\footnotesize\bfseries}r | >{\footnotesize}p{0.8\textwidth}}
        20xx & \emph{\lorem}, Periódico humorístico insitucional O Valonguiano. \\
        2022 & \emph{Computation for Architecture, hybrid visual and textual language: Research developments and considerations about the implementation of structural imperative and object-oriented paradigms.}, International Journal Of Architectural Computing. \\
        2021 & \emph{Designing Learning Methods: Programming with Visual and Textual Language in Python.}, XXV International Conference of the Iberoamerican Society of Digital Graphics. \\
        2016 & \emph{Desenvolvimento de um Sistema Portátil Multiaplicativo de Fluoresscência de Raios X}, Seminário Latino-americano de Análises por Técnicas de Raios X. \\

\end{tabular}

\end{minipage}\hfill



\end{paracol}

\end{document}
